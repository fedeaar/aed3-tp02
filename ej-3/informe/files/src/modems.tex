% el problema
El problema de los \textit{módems} que consideraremos tiene la siguiente premisa. Dado un conjunto de $N$ oficinas 
\begin{equation*}
    O := \{o_1\ ...\ o_N\}    
\end{equation*}
donde, para cada $1 \leq i \leq N$, la oficina $o_i$ se representa por su posición $(x_i,\ y_i)$ en el plano cartesiano ---con unidad en centímetros---, queremos encontrar el costo mínimo a emplear en cables UTP y en cables de fibra óptica para conectar todas las oficinas a internet\footnote{Notar que una oficina tendrá acceso a internet si y sólo si tiene un módem o si está conectada a una oficina con acceso a internet.}.

Para ello, vamos a contar con $W < N$ módems y vamos a disponer de cables UTP y de fibra óptica, cuyo costo por centímetro será $U$ y $V$ pesos, respectivamente. Dado que los cables UTP tienen ciertas restricciones, estos se podrán utilizar si y sólo si la distancia entre dos oficinas es menor a $R$ centímetros. 

Por ejemplo, si
\begin{equation*}
    O = \{(0,\ 0),\ (0,\ 1),\ (1,\ 0)\},\ R = 2,\ W = 1,\ U = 1\ \: \text{y}\ \: V = 1  
\end{equation*} 
luego una solución puede situar un módem en la oficina $(0,\ 0)$ y conectarla a cada una de las restantes con un centímetro de cable UTP, por un costo total de dos pesos.

% modelado
\subsection{Modelado como un problema de árbol generador mínimo}

Dada la descripción anterior, vamos a demostrar que el problema se puede modelar como una variante del problema del árbol generador mínimo. Lo detallamos a continuación.

Sea $G$ un grafo pesado completo cuyos vértices son el conjunto de oficinas $O$. Por cada par de vértices $o_i,\ o_j \in O$, definimos el costo de la arista $(o_i,\ o_j)$ como
\begin{equation*}
    w(o_i,\ o_j) = \begin{cases}
        d_{ij} \cdot U &\text{si}\ U \leq V\ \: \text{y}\ \: d_{ij} < R \\
        d_{ij} \cdot V & \text{si no}
    \end{cases}
\end{equation*}
donde $d_{ij}$ es la distancia euclideana entre ambas oficinas. 

Luego, de haber, al menos, una oficina con un módem, $G$ modela una solución no mínima en la que todas las oficinas están conectadas entre si con la opción más barata disponible entre un cable UTP y uno de fibra óptica. Si $W = 1$, sigue que, para mantener a todas las oficinas conectadas y minimizar el costo empleado en los cables, basta encontrar un árbol generador mínimo de $G$. 

Sin embargo, si $W > 1$, podemos reducir aún más el costo si, en vez de encontrar un árbol, encontramos un bosque de $W$ componentes conexas cuyo peso sea mínimo.

Notar que este bosque se puede generar removiendo las $W-1$ aristas de peso máximo de un árbol generador mínimo $T$ de $G$. Esto se debe a que esta operación genera un grafo de $N - W$ aristas, por lo que se compone de $W$ componentes conexas que son árboles y tiene peso mínimo ya que $T$ es mínimo y estamos removiendo sus aristas de peso máximo. 

Vamos a demostrar que lo podemos encontrar  modificando el algoritmo de \textit{Kruskal} para que termine en la iteración $N-W$.

% correctitud
\subsection{Demostración de correctitud}


% algoritmo
\subsection{El algoritmo}

Para ello, proponemos el siguiente algoritmo.


% complejidad
\subsection{Complejidad temporal y espacial}