En la teoría de grafos, el problema del \textit{árbol generador mínimo}\footnote{ Ver Thomas H. Cormen; Charles E. Leiserson; Ronald L. Rivest y Clifford Stein. Introduction to algorithms. 2009. Sección 23: \textit{Minimum Spanning Trees}.\label{foot_1}} ---o \textit{minimum spanning tree}, en inglés--- se refiere al problema de encontrar, para un grafo conexo \mbox{$G = (V,\ E)$} con función de peso $w : E \to \mathbb{R}$ asociada, un subgrafo conexo y acíclico de $G$ ---es decir, un árbol generador--- que minimice la suma total del peso de sus aristas.

Existen diversos métodos para la resolución de este problema. Entre ellos, los algoritmos \textit{golosos} de \textit{Prim} y de \textit{Kruskal}, que se basan en la selección de aristas de peso mínimo \textit{seguras}\footnote{ Una arista es \textit{segura} si y sólo si al agregarla a un subgrafo de un árbol generador, el resultado sigue siendo un subgrafo de algún árbol generador.} para la construcción de una solución.  

El siguiente informe evalúa el problema de los \textit{módems}, explicado en el apartado siguiente, y lo reformula como una variante del problema de \textit{árbol generador mínimo} que aprovecha el invariante del algoritmo de \textit{Kruskal}. Además, evalúa la eficiencia de la solución propuesta de manera empírica en función de la aplicación de diferentes estructuras de datos: \textit{priority queue} y \textit{disjoint set}; y heurísticas: \textit{union by rank} y \textit{path compression}\footnote{ Ver nota al pie (\ref{foot_1}). Sección 21: \textit{Data Structures for Disjoint Sets}.}.

$\\$
\noindent Palabras clave: \textit{árbol generador mínimo, algoritmo de Kruskal}.
